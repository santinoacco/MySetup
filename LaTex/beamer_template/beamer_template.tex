% Latex template: mahmoud.s.fahmy@students.kasralainy.edu.eg
% For more details: https://www.sharelatex.com/learn/Beamer

%\documentclass[8pt]{beamer}					% Document class
%\documentclass[aspectratio=169,8pt]{beamer}					% Document class
\documentclass[aspectratio=1610,8pt]{beamer}					% Document class

% ==== PACKAGES ==== %
\usepackage[english]{babel}				% Set language
\usepackage[utf8x]{inputenc}			% Set encoding
\usepackage[T1]{fontenc}
%\usepackage{tgbonum}

%\usepackage{fontspec}
%\setmainfot{Sanz}

\mode<presentation>						% Set options
{%
    % -- Set theme
    %\usetheme{default}					
    \usetheme{Singapore}					% Set theme
    %\usetheme[progressbar=frametitle]{Singapore}					% Set theme
    %\usetheme{Boadilla}					% Set theme
    % -- Set colors
    %\usecolortheme{default} 				
    \usecolortheme{rose} 				% Set colors
    % --  Set font theme
    %\usefonttheme{structureitalicserif}  				
    %\usefonttheme{structuresmallcapsserif}  				% Set font theme
    \usefonttheme{serif}  				% Set font theme
    \setbeamertemplate{caption}[numbered]	% Set caption to be numbered
    \setbeamertemplate{footline}[frame number]
}

% Uncomment this to have the outline at the beginning of each section highlighted.
%\AtBeginSection[]
%{
%  \begin{frame}{Outline}
%    \tableofcontents[currentsection]
%  \end{frame}
%}
\usepackage[export]{adjustbox}
% -- For including figures
\usepackage{graphicx}					
% -- For table rules
\usepackage{booktabs}					
% -- For cross-referencing
\usepackage{hyperref}					
\hypersetup{%
    colorlinks=true,
    linkcolor=blue,
    %filecolor=magenta,      
    filecolor=grey,      
    urlcolor=cyan,
}
% -- To Add code
\usepackage{verbatim}
\usepackage{listings}
% -- For math eq
\usepackage{amsmath}
% -- Subfigures
\usepackage{caption}
\usepackage{subcaption}
\usepackage{eso-pic}
%\newcommand\AtPagemyUpperLeft[1]{\AtPageLowerLeft{%
%\put(\LenToUnit{0.9\paperwidth},\LenToUnit{0.9\paperheight}){#1}}}

\title{%
    Performance of photon triggers defined in the Run-3 Menu.
    %CNN for photon ID at HLT
    %Qualification Task Update \\
    %\small{subtitle}
}
\author{\underline{S.~Noacco~Rosende}, F.~Monticelli}
%\author[Noacco, Monticelli]{S.~Noacco~Rosende\inst{1}, F.~Monticelli\inst{1}}
\institute[VFU]{%
    \inst{1}
    Instituto de F\'isica La Plata (CONICET-UNLP) \\
    \includegraphics[height=1cm,width=.8cm,keepaspectratio]{images/iflp_logo.png}
    \and
    \inst{2}
    Universidad Nacional de La Plata \\
    \includegraphics[height=1cm,width=2cm,keepaspectratio]{images/unlp_logo.png}
    %\includegraphics[height=1cm,width=.8cm,keepaspectratio]{images/iflp_logo.png}
}				
\date{\today}									% Today's date	

% -- Put logos on top right
\usepackage{textpos}
\addtobeamertemplate{frametitle}{}{%
    \begin{textblock*}{100mm}(.9\textwidth,-.8cm)
    \includegraphics[height=.7cm,width=2cm,keepaspectratio]{images/unlp_logo.png}
    \end{textblock*}
    \begin{textblock*}{100mm}(.8\textwidth,-.75cm)
    \includegraphics[height=.8cm,width=.8cm,keepaspectratio]{images/iflp_logo.png}
    \end{textblock*}
}
%\logo{%
    %\includegraphics[height=0.5cm]{images/unlp_logo.png}
    %\includegraphics[height=0.5cm]{images/iflp_logo.png}
%}

\begin{document}
%  -- Title page
\begin{frame}
  \titlepage
\end{frame}

% -- Outline
% This page includes the outline (Table of content) of the presentation.
% All sections and subsections will appear in the outline by default.
%\begin{frame}{Outline}
  %\tableofcontents
%\end{frame}

\section{Introduction}
\begin{frame}{Context}
    \begin{block}{QualificationTask resume}
        \textit{%
            This qualification task aims to setup and measure the performance of photon triggers defined in the Run-3 Menu.
            The work will include use the Offline monitoring framework to measure the efficiencies w.r.t.\ offline identification,
            modify, fix and maintain its configuration if needed,
            %and use official Z-radiative decay framework.
            \\
            In this task,
            \textbf{performance of Run-3 photon chains will be compared with results from Run 2}.
            In addition, the performance of alternative selections if/when available 
            (such as NN based identification at fast reconstruction level) will be evaluated.
            }
    \end{block}
    
    \begin{block}{Previously}
        \begin{itemize}
            \item [-]
                I showed a comparison between Run3 triggers on a Signal sample and Run2 triggers on an Enhanced Bias sample.
            \item [-]
                I showed Run3 triggers Efficiencies and Resolutions at HLT level.
        \end{itemize}
    \end{block}

    \begin{block}{Todays topic}
        I will compare the Efficiencies (at every step) and Resolutions (at HLT) between the triggers defined in Run2 against those of Run3 using the TrigMonitoring Tool,
        for Signal samples.
        \begin{itemize}
        \item[-]
            Signal (\textbf{HGam}) sample for Run2 data: \
            \texttt{\tiny{mc16\_13TeV.345306.PowhegHerwig7EvtGen\_NNLOPS\_nnlo\_30\_ggH125\_gamgam.merge.AOD.e5853\_e5984\_s3126\_r10724\_r10726}}
        \item[-]
            Signal (\textbf{HGam}) samples from the \href{https://its.cern.ch/jira/browse/ATR-22875}{11th ATHENA validation sampleA} for Run3.
            \begin{itemize}
            \item[o]
                with pileup: \ 
                \texttt{\tiny{user.okumura.val.run3DQ.343981.e5607\_e5984\_s3126\_d1616\_r12400\_tid24243105\_00.p1\_v3\_EXT0}} \
            \item[o]
                and without pileup: \ 
                \texttt{\tiny{user.okumura.val.run3DQ.343981.e5607\_e5984\_s3126f\_d1617\_r12400\_tid24243116\_00.p1\_v3\_EXT0}} \
            \end{itemize}
        \end{itemize}
    \end{block}
\end{frame}

%\begin{frame}{Package Overview}
    %\begin{block}{~}
        %For this QT I developed a SWAN package \textit{TriggerMonitoring~Visualization~Tool} (TMVT) to help visualize,
    %analize and debug the photon trigger performance from the output of the \textit{TrigMonitoring Tool} \texttt{Run3DQTestingDriver}.
    %The TMVT can be found in GitLab \href{https://gitlab.cern.ch/snoaccor/triggerMonitoring_visualizationTool}{here} is powered by Python and many \href{https://github.com/scikit-hep}{Scikit-HEP Tools}.
    %\end{block}
    %Some of its perks are:
    %\begin{itemize}
        %\begin{columns}
        %\column{0.3\textwidth}
        %\item \textit{\textcolor{blue}{Interactive search}} \\
            %Much like a \texttt{TBrowser} object to scope through your file trying to figure out the dir structure or the name of your relevant variables,
            %the \texttt{IO.explore\_Tfile} method interactively goes through the ROOT file until you find a \texttt{TH1F} o\texttt{TProfile} object.
        %\column{0.7\textwidth}
        %\begin{figure}[h!]
            %\includegraphics[width=.7\textwidth]{../TrigMonitoring/images/SshotExploreTFileInteractive.png}
        %\end{figure}
        %\end{columns}
    %\end{itemize}
%\end{frame}

%\begin{frame}{Package Overview}
    %\begin{block}{~}
    %The following results were computed using the \textit{TrigMonitoring Tool} \texttt{Run3DQTestingDriver} and plotted using the \textit{TriggerMonitoring~Visualization~Tool} which can be found in GitLab \href{https://gitlab.cern.ch/snoaccor/triggerMonitoring_visualizationTool}{here.}
    %Powered by Python and many \href{https://github.com/scikit-hep}{Scikit-HEP Tools},
    %this SWAN package is aimed to help visualize and debug the performance of the TrigMonitoring Tool.
    %\end{block}
    %Some of its perks are:
    %\begin{itemize}
        %\begin{columns}
        %\column{0.3\textwidth}
        %\item \textit{\textcolor{blue}{Show empties}} \\
            %It is straight forward to filter the desired objects,
            %as well as checking which of those are empty.
        %\column{0.7\textwidth}
        %\begin{figure}[h!]
            %\includegraphics[width=.6\textwidth]{../TrigMonitoring/images/SshotShowEmpties.png}
            %%\includegraphics[width=6cm,height=6cm]{../TrigMonitoring/images/SshotShowEmpties.png}
        %\end{figure}
        %\end{columns}
    %\end{itemize}
%\end{frame}

%\begin{frame}{Package Overview}
    %\begin{block}{~}
    %The following results were computed using the \textit{TrigMonitoring Tool} \texttt{Run3DQTestingDriver} and plotted using the \textit{TriggerMonitoring~Visualization~Tool} which can be found in GitLab \href{https://gitlab.cern.ch/snoaccor/triggerMonitoring_visualizationTool}{here.}
    %Powered by Python and many \href{https://github.com/scikit-hep}{Scikit-HEP Tools},
    %this SWAN package is aimed to help visualize and debug the performance of the TrigMonitoring Tool.
    %\end{block}
    %Some of its perks are:
    %\begin{itemize}
        %\begin{columns}
        %\column{0.2\textwidth}
        %\item \textit{\textcolor{blue}{All at a glance}} \\
            %Finally if one wants to check what is inside some specific directory,
            %it is easy to plot everything by just defining a filter and using the \texttt{plotter.show\_histos} method
        %\column{0.8\textwidth}
        %\begin{figure}[h!]
            %%\includegraphics[width=6cm,height=5cm,keepaspectratio]{../TrigMonitoring/images/SshotPlotAll.png}
            %\includegraphics[width=6cm,height=5cm]{../TrigMonitoring/images/SshotPlotAll.png}
        %\end{figure}
        %\end{columns}
    %\end{itemize}
%\end{frame}

%\begin{frame}{Results Overview}
    %In the following slides,
    %the folowing results will be shown.
    %All plots shown using TMVT.
    %\begin{itemize}
        %\item $npvtx$ distributions
        %\begin{itemize}
            %\item[-]
                %g20: Loose, Medium, Tight, Tight ICaloLoose, Tight ICaloMedium
            %\item[-]
                %g22: Tight
            %\item[-]
                %g25: Loose, Medium, Tight
            %\item[-]
                %g120: Loose
            %\item[-]
                %g140: Loose
        %\end{itemize}
        %\item Run2 vs Run3
        %\begin{itemize}
            %\item[-]
                %g120 Loose WP
            %\item[-]
                %g20 Tight ICaloLoose WP
            %\item[-]
                %g25 Medium WP
        %\end{itemize}
        %\item Run3 efficiencies
        %\begin{itemize}
            %\item[-]
                %g20: Loose, Medium, Tight, Tight ICaloLoose, Tight ICaloMedium
            %\item[-]
                %g22: Tight
            %\item[-]
                %g25: Loose, Medium, Tight
            %\item[-]
                %g120: Loose
        %\end{itemize}
        %\item Run3 resolutions
        %\begin{itemize}
            %\item [-]
                %Eratio
            %\item [-]
                %Reta
            %\item [-]
                %ethad
            %\item [-]
                %eta
            %\item [-]
                %Rhad
        %\end{itemize}
    %\end{itemize}
%\end{frame}

%\begin{frame}{Samples number of primary vertex distributions}
%\framesubtitle{g140}
    %\begin{figure}[h!]
        %\centering
        %\begin{subfigure}[b]{.45\textwidth}
            %\centering
            %\includegraphics[width=\textwidth]{../TrigMonitoring/images/outputs/Dist_Run3 No PileUp_g140*VH_npvtx.png}
            %\caption{~}
        %\end{subfigure}
        %\hfill
        %\begin{subfigure}[b]{.45\textwidth}
            %\centering
            %\includegraphics[width=\textwidth]{../TrigMonitoring/images/outputs/Dist_Run3 With PileUp_g140*VH_npvtx.png}
            %\caption{~}
        %\end{subfigure}
        %\hfill
    %\end{figure}
%\end{frame}

%\begin{frame}{Samples number of primary vertex distributions}
%\framesubtitle{g120}
    %\begin{figure}[h!]
        %\centering
        %\begin{subfigure}[b]{.45\textwidth}
            %\centering
            %\includegraphics[width=\textwidth]{../TrigMonitoring/images/outputs/Dist_Run3 No PileUp_g120*VHI_npvtx.png}
            %\caption{~}
        %\end{subfigure}
        %\hfill
        %\begin{subfigure}[b]{.45\textwidth}
            %\centering
            %\includegraphics[width=\textwidth]{../TrigMonitoring/images/outputs/Dist_Run3 With PileUp_g120*VHI_npvtx.png}
            %\caption{~}
        %\end{subfigure}
        %\hfill
    %\end{figure}
%\end{frame}

%\begin{frame}{Samples number of primary vertex distributions}
%\framesubtitle{g25}
    %\begin{figure}[h!]
        %\centering
        %\begin{subfigure}[b]{.45\textwidth}
            %\centering
            %\includegraphics[width=\textwidth]{../TrigMonitoring/images/outputs/Dist_Run3 No PileUp_g25*VH_npvtx.png}
            %\caption{~}
        %\end{subfigure}
        %\hfill
        %\begin{subfigure}[b]{.45\textwidth}
            %\centering
            %\includegraphics[width=\textwidth]{../TrigMonitoring/images/outputs/Dist_Run3 With PileUp_g25*VH_npvtx.png}
            %\caption{~}
        %\end{subfigure}
        %\hfill
    %\end{figure}
%\end{frame}

%\begin{frame}{Samples number of primary vertex distributions}
%\framesubtitle{g22}
    %\begin{figure}[h!]
        %\centering
        %\begin{subfigure}[b]{.45\textwidth}
            %\centering
            %\includegraphics[width=\textwidth]{../TrigMonitoring/images/outputs/Dist_Run3 No PileUp_g22*VH_npvtx.png}
            %\caption{~}
        %\end{subfigure}
        %\hfill
        %\begin{subfigure}[b]{.45\textwidth}
            %\centering
            %\includegraphics[width=\textwidth]{../TrigMonitoring/images/outputs/Dist_Run3 With PileUp_g22*VH_npvtx.png}
            %\caption{~}
        %\end{subfigure}
        %\hfill
    %\end{figure}
%\end{frame}

%\begin{frame}{Samples number of primary vertex distributions}
%\framesubtitle{g20}
    %\begin{figure}[h!]
        %\centering
        %\begin{subfigure}[b]{.45\textwidth}
            %\centering
            %\includegraphics[width=\textwidth]{../TrigMonitoring/images/outputs/Dist_Run3 No PileUp_g20*VH_npvtx.png}
            %\caption{~}
        %\end{subfigure}
        %\hfill
        %\begin{subfigure}[b]{.45\textwidth}
            %\centering
            %\includegraphics[width=\textwidth]{../TrigMonitoring/images/outputs/Dist_Run3 With PileUp_g20*VH_npvtx.png}
            %\caption{~}
        %\end{subfigure}
        %\hfill
        %\begin{subfigure}[b]{.45\textwidth}
            %\centering
            %\includegraphics[width=\textwidth]{../TrigMonitoring/images/outputs/Dist_Run3 No PileUp_g20*VHI_npvtx.png}
            %\caption{~}
        %\end{subfigure}
        %\hfill
        %\begin{subfigure}[b]{.45\textwidth}
            %\centering
            %\includegraphics[width=\textwidth]{../TrigMonitoring/images/outputs/Dist_Run3 With PileUp_g20*VHI_npvtx.png}
            %\caption{~}
        %\end{subfigure}
        %\hfill
    %\end{figure}
%\end{frame}
\section{Efficiencies}
\subsection{g20}
\begin{frame}{Run2 vs Run3 Trigger Efficiencies}
    Comparison between Run2 and Run3 triggers efficiencies for g20\_tight\_icaloloose\_L1M15VHI trigger at all levels in $\eta$.
    \begin{columns}
        %\column{width=9}
        \column{0.8\textwidth}
        %\hspace{-1}
        \begin{figure}[h!]
            \begin{subfigure}[b]{0.45\textwidth}
                \centering
                \includegraphics[width=0.9\linewidth]{../TrigMonitoring/images/outputs/Run2vsRun3/L1Calo_Eff_g20tighticaloloose_eta.png}
                %\includegraphics[width=5.5cm,height=4cm]{../TrigMonitoring/images/outputs/Run2vsRun3/L1Calo_Eff_g20tighticaloloose_eta.png}
                \caption{~}
            \end{subfigure}
            \begin{subfigure}[b]{0.45\textwidth}
                \centering
                \includegraphics[width=0.9\linewidth]{../TrigMonitoring/images/outputs/Run2vsRun3/EFCalo_Eff_g20tighticaloloose_eta.png}
                %\includegraphics[width=5.5cm,height=4cm]{../TrigMonitoring/images/outputs/Run2vsRun3/EFCalo_Eff_g20tighticaloloose_eta.png}
                \caption{~}
            \end{subfigure}
            \begin{subfigure}[b]{0.45\textwidth}
                \centering
                \includegraphics[width=0.9\linewidth]{../TrigMonitoring/images/outputs/Run2vsRun3/L2Calo_Eff_g20tighticaloloose_eta.png}
                %\includegraphics[width=5.5cm,height=4cm]{../TrigMonitoring/images/outputs/Run2vsRun3/L2Calo_Eff_g20tighticaloloose_eta.png}
                \caption{~}
            \end{subfigure}
            \begin{subfigure}[b]{0.45\textwidth}
                \centering
                \includegraphics[width=0.9\linewidth]{../TrigMonitoring/images/outputs/Run2vsRun3/HLT_Eff_g20tighticaloloose_eta.png}
                %\includegraphics[width=5.5cm,height=4cm]{../TrigMonitoring/images/outputs/Run2vsRun3/HLT_Eff_g20tighticaloloose_eta.png}
                \caption{~}
            \end{subfigure}
        \end{figure}

        \column{0.3\textwidth}
        \begin{itemize}
            \item [-] 
                Although we have the framework ready to show all steps of the trigger,
                these samples show the known bug of containing the HLT data for all steps.
            \item [-]
                There is a big discrepancy between Run3 and Run2 for $\eta$ in the endcap region. 
            \item [-]
                icaloloose could be the cause of this.
            \item [-]
                We should explore what is going on in the intermidiate steps,
                and for that we will need to get the data for those steps.
        \end{itemize}
    \end{columns}
\end{frame}

\begin{frame}{Run2 vs Run3 Trigger Efficiencies}
    Comparison between Run2 and Run3 triggers efficiencies vs $p_T$ for g20\_tight\_icaloloose\_L1M15VHI.
    \begin{columns}
        \column{0.6\textwidth}
        \begin{figure}[h!]
            \centering
            \includegraphics[width=0.9\linewidth]{../TrigMonitoring/images/outputs/Run2vsRun3/HLT_Eff_g20tighticaloloose_pt.png}
        \end{figure}
        \column{0.4\textwidth}
        \begin{itemize}
            \item [-]
                Run2 vs Run3 are comparable in the whole range.
            \item [-]
                We see a sharper turn on due to an improved $E_T$ resolution for Run3,\
                explained by the fact that HLT is implementing the same clustering reconstruction as offline (SuperClusters),\
                while Run2 implemented Sliding-Window.
        \end{itemize}
    \end{columns}
    
    %\begin{columns}
        %\column{0.8\textwidth}
        %\begin{figure}[h!]
            %\centering
            %\begin{subfigure}[b]{0.45\textwidth}
                %\centering
                %\includegraphics[width=0.9\linewidth]{../TrigMonitoring/images/outputs/Run2vsRun3/L1Calo_Eff_g20tighticaloloose_pt.png}
                %\includegraphics[width=5.5cm,height=4cm]{../TrigMonitoring/images/outputs/Run2vsRun3/L1Calo_Eff_g20tighticaloloose_eta.png}
                %\caption{~}
            %\end{subfigure}
            %\begin{subfigure}[b]{0.45\textwidth}
                %\centering
                %\includegraphics[width=0.9\linewidth]{../TrigMonitoring/images/outputs/Run2vsRun3/EFCalo_Eff_g20tighticaloloose_pt.png}
                %\includegraphics[width=5.5cm,height=4cm]{../TrigMonitoring/images/outputs/Run2vsRun3/EFCalo_Eff_g20tighticaloloose_eta.png}
                %\caption{~}
            %\end{subfigure}
            %\begin{subfigure}[b]{0.45\textwidth}
                %\centering
                %\includegraphics[width=0.9\linewidth]{../TrigMonitoring/images/outputs/Run2vsRun3/L2Calo_Eff_g20tighticaloloose_pt.png}
                %\includegraphics[width=5.5cm,height=4cm]{../TrigMonitoring/images/outputs/Run2vsRun3/L2Calo_Eff_g20tighticaloloose_eta.png}
                %\caption{~}
            %\end{subfigure}
            %\begin{subfigure}[b]{0.45\textwidth}
                %\centering
                %\includegraphics[width=0.9\linewidth]{../TrigMonitoring/images/outputs/Run2vsRun3/HLT_Eff_g20tighticaloloose_pt.png}
                %\includegraphics[width=5.5cm,height=4cm]{../TrigMonitoring/images/outputs/Run2vsRun3/HLT_Eff_g20tighticaloloose_eta.png}
                %\caption{~}
            %\end{subfigure}
        %\end{figure}

        %\column{0.3\textwidth}
        %\begin{itemize}
            %\item [-] 
                %Run3 turn on is sharper due to improv et res,
                %given using same recong algo like offline (supercluster), while run2 is (sliding-window.)
        %\end{itemize}
    %\end{columns}
\end{frame}

\subsection{g25}
\begin{frame}{Run2 vs Run3 Trigger Efficiencies}
    Comparison between Run2 and Run3 triggers efficiencies in $\eta$ (left) and $p_T$ (right) for g25\_medium\_L1EM20VH.
    \begin{columns}
        \column{0.9\textwidth}
        \begin{figure}[h!]
            \centering
            \begin{subfigure}[b]{.45\textwidth}
                \centering
                \includegraphics[width=\textwidth]{../TrigMonitoring/images/outputs/Run2vsRun3/HLT_Eff_g25medium_eta.png}
                \caption{~}
            \end{subfigure}
            \hfill
            \begin{subfigure}[b]{.45\textwidth}
                \centering
                \includegraphics[width=\textwidth]{../TrigMonitoring/images/outputs/Run2vsRun3/HLT_Eff_g25medium_pt.png}
                \caption{~}
            \end{subfigure}
            \hfill
        \end{figure}

        %\column{0.3\textwidth}
        \begin{itemize}
            \item [-]
                For $ \eta$ we see again a big difference of performance between Run2 and Run3 in the endcap region,\
                the latter performs worse.
            \item [-]
                For $p_T$ we see a worse performance of the Run3 trigger against Run2 over the whole range,\
                but the turn on is sharper for the first one.
        \end{itemize}
    \end{columns}
    %\begin{columns}
        %\column{0.8\textwidth}
        %\begin{figure}[h!]
            %%\centering
            %\begin{subfigure}[b]{0.45\textwidth}
                %\centering
                %\includegraphics[width=0.9\linewidth]{../TrigMonitoring/images/outputs/Run2vsRun3/L1Calo_Eff_g25medium_eta.png}
                %%\includegraphics[width=5.5cm,height=4cm]{../TrigMonitoring/images/outputs/Run2vsRun3/L1Calo_Eff_g20tighticaloloose_eta.png}
                %\caption{~}
            %\end{subfigure}
            %\begin{subfigure}[b]{0.45\textwidth}
                %\centering
                %\includegraphics[width=.9\linewidth]{../TrigMonitoring/images/outputs/Run2vsRun3/EFCalo_Eff_g25medium_eta.png}
                %\caption{~}
            %\end{subfigure}
            %\begin{subfigure}[b]{0.45\textwidth}
                %\centering
                %\includegraphics[width=0.9\linewidth]{../TrigMonitoring/images/outputs/Run2vsRun3/L2Calo_Eff_g25medium_eta.png}
                %\caption{~}
            %\end{subfigure}
            %\begin{subfigure}[b]{0.45\textwidth}
                %\centering
                %\includegraphics[width=0.9\linewidth]{../TrigMonitoring/images/outputs/Run2vsRun3/HLT_Eff_g25medium_eta.png}
                %\caption{~}
            %\end{subfigure}
        %\end{figure}

        %\column{0.3\textwidth}
        %\begin{itemize}
            %\item [-] 
        %\end{itemize}
    %\end{columns}
\end{frame}
%\begin{frame}{Run2 vs Run3 Trigger Efficiencies}
    %Comparison between Run2 and Run3 triggers efficiencies for all g25 trigger levels in $p_T$.
    %\begin{columns}
        %\column{0.8\textwidth}
        %\begin{figure}[h!]
            %%\centering
            %\begin{subfigure}[b]{0.45\textwidth}
                %\centering
                %\includegraphics[width=0.9\linewidth]{../TrigMonitoring/images/outputs/Run2vsRun3/L1Calo_Eff_g25medium_pt.png}
                %\caption{~}
            %\end{subfigure}
            %\begin{subfigure}[b]{0.45\textwidth}
                %\centering
                %\includegraphics[width=.9\linewidth]{../TrigMonitoring/images/outputs/Run2vsRun3/EFCalo_Eff_g25medium_pt.png}
                %\caption{~}
            %\end{subfigure}
            %\begin{subfigure}[b]{0.45\textwidth}
                %\centering
                %\includegraphics[width=0.9\linewidth]{../TrigMonitoring/images/outputs/Run2vsRun3/L2Calo_Eff_g25medium_pt.png}
                %\caption{~}
            %\end{subfigure}
            %\begin{subfigure}[b]{0.45\textwidth}
                %\centering
                %\includegraphics[width=0.9\linewidth]{../TrigMonitoring/images/outputs/Run2vsRun3/HLT_Eff_g25medium_pt.png}
                %\caption{~}
            %\end{subfigure}
        %\end{figure}

        %\column{0.3\textwidth}
        %\begin{itemize}
            %\item [-] 
        %\end{itemize}

    %\end{columns}
%\end{frame}

\subsection{g120}
\begin{frame}{Run2 vs Run3 Trigger Efficiencies}
    Comparison between Run2 and Run3 triggers efficiencies in $\eta$ (left) and $p_T$ (right) for g120\_loose\_L1EM22VHI.
    \begin{columns}
        \column{0.9\textwidth}
        \begin{figure}[h!]
            \centering
            \begin{subfigure}[b]{.45\textwidth}
                \centering
                \includegraphics[width=\textwidth]{../TrigMonitoring/images/outputs/Run2vsRun3/HLT_Eff_g120loose_eta.png}
                \caption{~}
            \end{subfigure}
            \hfill
            \begin{subfigure}[b]{.45\textwidth}
                \centering
                \includegraphics[width=\textwidth]{../TrigMonitoring/images/outputs/Run2vsRun3/HLT_Eff_g120loose_highet.png}
                \caption{~}
            \end{subfigure}
            \hfill
        \end{figure}
        
        %\column{0.3\textwidth}
        \begin{itemize}
            \item [-]
                For $ \eta$ we see a worse performance from the Run3 trigger against the Run2 trigger over the whole range and both with and without Pileup. 
           \item [-]
                For $p_T$ the performance is also significantly worse for the Run3 trigger against the Run2 trigger over the whole range up to $p_T \ge 225$ $GeV$
        \end{itemize}
    \end{columns}
    %\begin{columns}
        %\column{0.8\textwidth}
        %\begin{figure}[h!]
            %%\centering
            %\begin{subfigure}[b]{.45\textwidth}
                %\centering
                %\includegraphics[width=.9\linewidth]{../TrigMonitoring/images/outputs/Run2vsRun3/L1Calo_Eff_g120loose_eta.png}
                %\caption{~}
            %\end{subfigure}
            %\begin{subfigure}[b]{.45\textwidth}
                %\centering
                %\includegraphics[width=.9\linewidth]{../TrigMonitoring/images/outputs/Run2vsRun3/EFCalo_Eff_g120loose_eta.png}
                %\caption{~}
            %\end{subfigure}
            %\begin{subfigure}[b]{.45\textwidth}
                %\centering
                %\includegraphics[width=0.9\linewidth]{../TrigMonitoring/images/outputs/Run2vsRun3/L2Calo_Eff_g120loose_eta.png}
                %\caption{~}
            %\end{subfigure}
            %\begin{subfigure}[b]{.45\textwidth}
                %\centering
                %\includegraphics[width=0.9\linewidth]{../TrigMonitoring/images/outputs/Run2vsRun3/HLT_Eff_g120loose_eta.png}
                %\caption{~}
            %\end{subfigure}
        %\end{figure}

        %\column{0.3\textwidth}
        %\begin{itemize}
            %\item [-] 
        %\end{itemize}
    %\end{columns}
\end{frame}
%\begin{frame}{Run2 vs Run3 Trigger Efficiencies}
    %Comparison between Run2 and Run3 triggers efficiencies for all g25 trigger levels in $p_T$.
    %\begin{columns}
        %\column{0.8\textwidth}
        %\begin{figure}[h!]
            %%\centering
            %\begin{subfigure}[b]{0.45\textwidth}
                %\centering
                %\includegraphics[width=0.9\linewidth]{../TrigMonitoring/images/outputs/Run2vsRun3/L1Calo_Eff_g120loose_highet.png}
                %\caption{~}
            %\end{subfigure}
            %\begin{subfigure}[b]{0.45\textwidth}
                %\centering
                %\includegraphics[width=.9\linewidth]{../TrigMonitoring/images/outputs/Run2vsRun3/EFCalo_Eff_g120loose_highet.png}
                %\caption{~}
            %\end{subfigure}
            %\begin{subfigure}[b]{0.45\textwidth}
                %\centering
                %\includegraphics[width=0.9\linewidth]{../TrigMonitoring/images/outputs/Run2vsRun3/L2Calo_Eff_g120loose_highet.png}
                %\caption{~}
            %\end{subfigure}
            %\begin{subfigure}[b]{0.45\textwidth}
                %\centering
                %\includegraphics[width=0.9\linewidth]{../TrigMonitoring/images/outputs/Run2vsRun3/HLT_Eff_g120loose_highet.png}
                %\caption{~}
            %\end{subfigure}
        %\end{figure}

        %\column{0.3\textwidth}
        %\begin{itemize}
            %\item [-] 
                %check if not missing evs in a previous level.
        %\end{itemize}

    %\end{columns}
%\end{frame}

%\begin{frame}{Run2 vs Run3 Trigger Efficiencies}
    %Comparison between Run2 and Run3 triggers efficiencies for all g120 trigger levels in $\eta$(left) and $p_T$(right).
    %\begin{figure}[h!]
        %\centering
        %\begin{subfigure}[b]{.45\textwidth}
            %\centering
            %\includegraphics[width=\textwidth]{../TrigMonitoring/images/outputs/Run2vsRun3/Eff_g120_loose_eta.png}
            %\caption{~}
        %\end{subfigure}
        %\hfill
        %\begin{subfigure}[b]{.45\textwidth}
            %\centering
            %\includegraphics[width=\textwidth]{../TrigMonitoring/images/outputs/Run2vsRun3/Eff_g120_loose_highet.png}
            %\caption{~}
        %\end{subfigure}
        %\hfill
    %\end{figure}
%\end{frame}

%\begin{frame}{Run3 Efficiencies}
    %\framesubtitle{g120}
    %No Pileup (top) and With Pileup (bottom) g120 trigger levels efficiencies in $\eta$(left), $\E_T$(middle) and $npvtx$(right).
    %\begin{figure}[h!]
        %\centering
        %\begin{subfigure}[b]{.3\textwidth}
            %\centering
            %\includegraphics[width=\textwidth]{../TrigMonitoring/images/outputs/Eff_Run3 No PileUp_g120*VHI_eta_passed_vs_eta.png}
            %\caption{~}
        %\end{subfigure}
        %%\hfill
        %\begin{subfigure}[b]{.3\textwidth}
            %\centering
            %\includegraphics[width=\textwidth]{../TrigMonitoring/images/outputs/Eff_Run3 No PileUp_g120*VHI_highet_passed_vs_highet.png}
            %\caption{~}
        %\end{subfigure}
        %%\hfill
        %\begin{subfigure}[b]{.3\textwidth}
            %\centering
            %\includegraphics[width=\textwidth]{../TrigMonitoring/images/outputs/Eff_Run3 No PileUp_g120*VHI_npvtx_passed_vs_npvtx.png}
            %\caption{~}
        %\end{subfigure}
        %%\hfill
        %\begin{subfigure}[b]{.3\textwidth}
            %\centering
            %\includegraphics[width=\textwidth]{../TrigMonitoring/images/outputs/Eff_Run3 With PileUp_g120*VHI_eta_passed_vs_eta.png}
            %\caption{~}
        %\end{subfigure}
        %%\hfill
        %\begin{subfigure}[b]{.3\textwidth}
            %\centering
            %\includegraphics[width=\textwidth]{../TrigMonitoring/images/outputs/Eff_Run3 With PileUp_g120*VHI_highet_passed_vs_highet.png}
            %\caption{~}
        %\end{subfigure}
        %%\hfill
        %\begin{subfigure}[b]{.3\textwidth}
            %\centering
            %\includegraphics[width=\textwidth]{../TrigMonitoring/images/outputs/Eff_Run3 With PileUp_g120*VHI_npvtx_passed_vs_npvtx.png}
            %\caption{~}
        %\end{subfigure}
        %%\hfill
    %\end{figure}
%\end{frame}

%\begin{frame}{Run3 Efficiencies}
    %\framesubtitle{g25}
    %No Pileup (top) and With Pileup (bottom) g25 trigger levels efficiencies in $\eta$(left), $\E_T$(middle) and $npvtx$(right).
    %\begin{figure}[h!]
        %\centering
        %\begin{subfigure}[b]{.3\textwidth}
            %\centering
            %\includegraphics[width=\textwidth]{../TrigMonitoring/images/outputs/Eff_Run3 No PileUp_g25*VH_pt_passed_vs_pt.png}
            %\caption{~}
        %\end{subfigure}
        %%\hfill
        %\begin{subfigure}[b]{.3\textwidth}
            %\centering
            %\includegraphics[width=\textwidth]{../TrigMonitoring/images/outputs/Eff_Run3 No PileUp_g25*VH_eta_passed_vs_eta.png}
            %\caption{~}
        %\end{subfigure}
        %%\hfill
        %\begin{subfigure}[b]{.3\textwidth}
            %\centering
            %\includegraphics[width=\textwidth]{../TrigMonitoring/images/outputs/Eff_Run3 No PileUp_g25*VH_npvtx_passed_vs_npvtx.png}
            %\caption{~}
        %\end{subfigure}
        %%\hfill
        %\begin{subfigure}[b]{.3\textwidth}
            %\centering
            %\includegraphics[width=\textwidth]{../TrigMonitoring/images/outputs/Eff_Run3 With PileUp_g25*VH_pt_passed_vs_pt.png}
            %\caption{~}
        %\end{subfigure}
        %%\hfill
        %\begin{subfigure}[b]{.3\textwidth}
            %\centering
            %\includegraphics[width=\textwidth]{../TrigMonitoring/images/outputs/Eff_Run3 With PileUp_g25*VH_eta_passed_vs_eta.png}
            %\caption{~}
        %\end{subfigure}
        %%\hfill
        %\begin{subfigure}[b]{.3\textwidth}
            %\centering
            %\includegraphics[width=\textwidth]{../TrigMonitoring/images/outputs/Eff_Run3 With PileUp_g25*VH_npvtx_passed_vs_npvtx.png}
            %\caption{~}
        %\end{subfigure}
        %%\hfill
    %\end{figure}
%\end{frame}

%\begin{frame}{Run3 Efficiencies}
    %\framesubtitle{g22}
    %No Pileup (top) and With Pileup (bottom) g22 trigger levels efficiencies in $\eta$(left), $\E_T$(middle) and $npvtx$(right).
    %\begin{figure}[h!]
        %\centering
        %\begin{subfigure}[b]{.3\textwidth}
            %\centering
            %\includegraphics[width=\textwidth]{../TrigMonitoring/images/outputs/Eff_Run3 No PileUp_g22*VH_pt_passed_vs_pt.png}
            %\caption{~}
        %\end{subfigure}
        %%\hfill
        %\begin{subfigure}[b]{.3\textwidth}
            %\centering
            %\includegraphics[width=\textwidth]{../TrigMonitoring/images/outputs/Eff_Run3 No PileUp_g22*VH_eta_passed_vs_eta.png}
            %\caption{~}
        %\end{subfigure}
        %%\hfill
        %\begin{subfigure}[b]{.3\textwidth}
            %\centering
            %\includegraphics[width=\textwidth]{../TrigMonitoring/images/outputs/Eff_Run3 No PileUp_g22*VH_npvtx_passed_vs_npvtx.png}
            %\caption{~}
        %\end{subfigure}
        %%\hfill
        %\begin{subfigure}[b]{.3\textwidth}
            %\centering
            %\includegraphics[width=\textwidth]{../TrigMonitoring/images/outputs/Eff_Run3 With PileUp_g22*VH_pt_passed_vs_pt.png}
            %\caption{~}
        %\end{subfigure}
        %%\hfill
        %\begin{subfigure}[b]{.3\textwidth}
            %\centering
            %\includegraphics[width=\textwidth]{../TrigMonitoring/images/outputs/Eff_Run3 With PileUp_g22*VH_eta_passed_vs_eta.png}
            %\caption{~}
        %\end{subfigure}
        %%\hfill
        %\begin{subfigure}[b]{.3\textwidth}
            %\centering
            %\includegraphics[width=\textwidth]{../TrigMonitoring/images/outputs/Eff_Run3 With PileUp_g22*VH_npvtx_passed_vs_npvtx.png}
            %\caption{~}
        %\end{subfigure}
        %%\hfill
    %\end{figure}
%\end{frame}

%\begin{frame}{Run3 Efficiencies}
    %\framesubtitle{g20}
    %No Pileup (top) and With Pileup (bottom) g20 trigger ``*VH'' levels efficiencies in $\eta$(left), $\E_T$(middle) and $npvtx$(right).
    %\begin{figure}[h!]
        %\centering
        %\begin{subfigure}[b]{.3\textwidth}
            %\centering
            %\includegraphics[width=\textwidth]{../TrigMonitoring/images/outputs/Eff_Run3 No PileUp_g20*VH_pt_passed_vs_pt.png}
            %\caption{~}
        %\end{subfigure}
        %%\hfill
        %\begin{subfigure}[b]{.3\textwidth}
            %\centering
            %\includegraphics[width=\textwidth]{../TrigMonitoring/images/outputs/Eff_Run3 No PileUp_g20*VH_eta_passed_vs_eta.png}
            %\caption{~}
        %\end{subfigure}
        %%\hfill
        %\begin{subfigure}[b]{.3\textwidth}
            %\centering
            %\includegraphics[width=\textwidth]{../TrigMonitoring/images/outputs/Eff_Run3 No PileUp_g20*VH_npvtx_passed_vs_npvtx.png}
            %\caption{~}
        %\end{subfigure}
        %%\hfill
        %\begin{subfigure}[b]{.3\textwidth}
            %\centering
            %\includegraphics[width=\textwidth]{../TrigMonitoring/images/outputs/Eff_Run3 With PileUp_g20*VH_pt_passed_vs_pt.png}
            %\caption{~}
        %\end{subfigure}
        %%\hfill
        %\begin{subfigure}[b]{.3\textwidth}
            %\centering
            %\includegraphics[width=\textwidth]{../TrigMonitoring/images/outputs/Eff_Run3 With PileUp_g20*VH_eta_passed_vs_eta.png}
            %\caption{~}
        %\end{subfigure}
        %%\hfill
        %\begin{subfigure}[b]{.3\textwidth}
            %\centering
            %\includegraphics[width=\textwidth]{../TrigMonitoring/images/outputs/Eff_Run3 With PileUp_g20*VH_npvtx_passed_vs_npvtx.png}
            %\caption{~}
        %\end{subfigure}
        %%\hfill
    %\end{figure}
%\end{frame}

%\begin{frame}{Run3 Efficiencies}
    %\framesubtitle{g20}
    %No Pileup (top) and With Pileup (bottom) g20 trigger ``*VHI'' levels efficiencies in $\eta$(left), $\E_T$(middle) and $npvtx$(right).
    %\begin{figure}[h!]
        %\centering
        %\begin{subfigure}[b]{.3\textwidth}
            %\centering
            %\includegraphics[width=\textwidth]{../TrigMonitoring/images/outputs/Eff_Run3 No PileUp_g20*VHI_pt_passed_vs_pt.png}
            %\caption{~}
        %\end{subfigure}
        %%\hfill
        %\begin{subfigure}[b]{.3\textwidth}
            %\centering
            %\includegraphics[width=\textwidth]{../TrigMonitoring/images/outputs/Eff_Run3 No PileUp_g20*VHI_eta_passed_vs_eta.png}
            %\caption{~}
        %\end{subfigure}
        %%\hfill
        %\begin{subfigure}[b]{.3\textwidth}
            %\centering
            %\includegraphics[width=\textwidth]{../TrigMonitoring/images/outputs/Eff_Run3 No PileUp_g20*VHI_npvtx_passed_vs_npvtx.png}
            %\caption{~}
        %\end{subfigure}
        %%\hfill
        %\begin{subfigure}[b]{.3\textwidth}
            %\centering
            %\includegraphics[width=\textwidth]{../TrigMonitoring/images/outputs/Eff_Run3 With PileUp_g20*VHI_pt_passed_vs_pt.png}
            %\caption{~}
        %\end{subfigure}
        %%\hfill
        %\begin{subfigure}[b]{.3\textwidth}
            %\centering
            %\includegraphics[width=\textwidth]{../TrigMonitoring/images/outputs/Eff_Run3 With PileUp_g20*VHI_eta_passed_vs_eta.png}
            %\caption{~}
        %\end{subfigure}
        %%\hfill
        %\begin{subfigure}[b]{.3\textwidth}
            %\centering
            %\includegraphics[width=\textwidth]{../TrigMonitoring/images/outputs/Eff_Run3 With PileUp_g20*VHI_npvtx_passed_vs_npvtx.png}
            %\caption{~}
        %\end{subfigure}
        %%\hfill
    %\end{figure}
%\end{frame}
\section{Resolutions}
%\subsection{g120}
%\subsection{g25}
%\subsection{g20}
\begin{frame}{Run2 vs Run3 Trigger Resolutions at HLT step}
    \framesubtitle{g20}
    g20 trigger levels resolutions for $Eratio$, $ethad$, $Rhad$ and $Reta$, $\eta$, $\phi$ from top left to bottom right.
    \begin{columns}
    \column{.7\textwidth}
    \begin{figure}[h!]
        \centering
        \begin{subfigure}[b]{.3\textwidth}
            \centering
            \includegraphics[width=\textwidth]{../TrigMonitoring/images/outputs/Run2vsRun3/Res_g20_tighticalo_res_eratio.png}
            \caption{~}
        \end{subfigure}
        %\hfill
        \begin{subfigure}[b]{.3\textwidth}
            \centering
            \includegraphics[width=\textwidth]{../TrigMonitoring/images/outputs/Run2vsRun3/Res_g20_tighticalo_res_ethad.png}
            \caption{~}
        \end{subfigure}
        %\hfill
        \begin{subfigure}[b]{.3\textwidth}
            \centering
            \includegraphics[width=\textwidth]{../TrigMonitoring/images/outputs/Run2vsRun3/Res_g20_tighticalo_res_Rhad.png}
            \caption{~}
        \end{subfigure}
        %\hfill
        \begin{subfigure}[b]{.3\textwidth}
            \centering
            \includegraphics[width=\textwidth]{../TrigMonitoring/images/outputs/Run2vsRun3/Res_g20_tighticalo_res_Reta.png}
            \caption{~}
        \end{subfigure}
        %\hfill
        \begin{subfigure}[b]{.3\textwidth}
            \centering
            \includegraphics[width=\textwidth]{../TrigMonitoring/images/outputs/Run2vsRun3/Res_g20_tighticalo_res_eta.png}
            \caption{~}
        \end{subfigure}
        %\hfill
        \begin{subfigure}[b]{.3\textwidth}
            \centering
            \includegraphics[width=\textwidth]{../TrigMonitoring/images/outputs/Run2vsRun3/Res_g20_tighticalo_res_phi.png}
            \caption{~}
        \end{subfigure}
    \end{figure}
    \column{.3\textwidth}
    Resolutions look alike from Run2 to Run3 with 2 notable exceptions:
    \begin{itemize}
        \item [-]
            $ \eta$ and $ \phi$ resolutions are significantly improved in Run3.
        \item [-]
            $R_{\eta}$ shows a right tail in Run3 which suggests there is some issue to be addressed.
        \item [-]
            $E_T$ resolutions empty for Run3.
    \end{itemize}
    \end{columns}
\end{frame}

%\begin{frame}{Resolutions}
    %\framesubtitle{g20 *VH - With PileUp}
    %g20 ``*VH'' trigger levels resolutions for $Eratio$, $\eta$, $ethad$, $Rhad$ and $Reta$ from top left to bottom right.
    %\begin{figure}[h!]
        %\centering
        %\begin{subfigure}[b]{.3\textwidth}
            %\centering
            %\includegraphics[width=\textwidth]{../TrigMonitoring/images/outputs/Res_Run 3 WithPileup_g20*VH_res_eratio.png}
            %\caption{~}
        %\end{subfigure}
        %%\hfill
        %\begin{subfigure}[b]{.3\textwidth}
            %\centering
            %\includegraphics[width=\textwidth]{../TrigMonitoring/images/outputs/Res_Run 3 WithPileup_g20*VH_res_eta.png}
            %\caption{~}
        %\end{subfigure}
        %%\hfill
        %\begin{subfigure}[b]{.3\textwidth}
            %\centering
            %\includegraphics[width=\textwidth]{../TrigMonitoring/images/outputs/Res_Run 3 WithPileup_g20*VH_res_ethad.png}
            %\caption{~}
        %\end{subfigure}
        %%\hfill
        %\begin{subfigure}[b]{.3\textwidth}
            %\centering
            %\includegraphics[width=\textwidth]{../TrigMonitoring/images/outputs/Res_Run 3 WithPileup_g20*VH_res_Rhad.png}
            %\caption{~}
        %\end{subfigure}
        %%\hfill
        %\begin{subfigure}[b]{.3\textwidth}
            %\centering
            %\includegraphics[width=\textwidth]{../TrigMonitoring/images/outputs/Res_Run 3 WithPileup_g20*VH_res_Reta.png}
            %\caption{~}
        %\end{subfigure}
        %%\hfill
    %\end{figure}
%\end{frame}

%\begin{frame}{Resolutions}
    %\framesubtitle{g20 *VHI - No PileUp}
    %g20 ``*VHI'' trigger levels resolutions for $Eratio$, $\eta$, $ethad$, $Rhad$ and $Reta$ from top left to bottom right.
    %\begin{figure}[h!]
        %\centering
        %\begin{subfigure}[b]{.3\textwidth}
            %\centering
            %\includegraphics[width=\textwidth]{../TrigMonitoring/images/outputs/Res_Run 3 NoPileup_g20*VHI_res_eratio.png}
            %\caption{~}
        %\end{subfigure}
        %%\hfill
        %\begin{subfigure}[b]{.3\textwidth}
            %\centering
            %\includegraphics[width=\textwidth]{../TrigMonitoring/images/outputs/Res_Run 3 NoPileup_g20*VHI_res_eta.png}
            %\caption{~}
        %\end{subfigure}
        %%\hfill
        %\begin{subfigure}[b]{.3\textwidth}
            %\centering
            %\includegraphics[width=\textwidth]{../TrigMonitoring/images/outputs/Res_Run 3 NoPileup_g20*VHI_res_ethad.png}
            %\caption{~}
        %\end{subfigure}
        %%\hfill
        %\begin{subfigure}[b]{.3\textwidth}
            %\centering
            %\includegraphics[width=\textwidth]{../TrigMonitoring/images/outputs/Res_Run 3 NoPileup_g20*VHI_res_Rhad.png}
            %\caption{~}
        %\end{subfigure}
        %%\hfill
        %\begin{subfigure}[b]{.3\textwidth}
            %\centering
            %\includegraphics[width=\textwidth]{../TrigMonitoring/images/outputs/Res_Run 3 NoPileup_g20*VHI_res_Reta.png}
            %\caption{~}
        %\end{subfigure}
        %%\hfill
    %\end{figure}
%\end{frame}

%\begin{frame}{Resolutions}
    %\framesubtitle{g20 *VHI - With PileUp}
    %g20 ``*VHI'' trigger levels resolutions for $Eratio$, $\eta$, $ethad$, $Rhad$ and $Reta$ from top left to bottom right.
    %\begin{figure}[h!]
        %\centering
        %\begin{subfigure}[b]{.3\textwidth}
            %\centering
            %\includegraphics[width=\textwidth]{../TrigMonitoring/images/outputs/Res_Run 3 WithPileup_g20*VHI_res_eratio.png}
            %\caption{~}
        %\end{subfigure}
        %%\hfill
        %\begin{subfigure}[b]{.3\textwidth}
            %\centering
            %\includegraphics[width=\textwidth]{../TrigMonitoring/images/outputs/Res_Run 3 WithPileup_g20*VHI_res_eta.png}
            %\caption{~}
        %\end{subfigure}
        %%\hfill
        %\begin{subfigure}[b]{.3\textwidth}
            %\centering
            %\includegraphics[width=\textwidth]{../TrigMonitoring/images/outputs/Res_Run 3 WithPileup_g20*VHI_res_ethad.png}
            %\caption{~}
        %\end{subfigure}
        %%\hfill
        %\begin{subfigure}[b]{.3\textwidth}
            %\centering
            %\includegraphics[width=\textwidth]{../TrigMonitoring/images/outputs/Res_Run 3 WithPileup_g20*VHI_res_Rhad.png}
            %\caption{~}
        %\end{subfigure}
        %%\hfill
        %\begin{subfigure}[b]{.3\textwidth}
            %\centering
            %\includegraphics[width=\textwidth]{../TrigMonitoring/images/outputs/Res_Run 3 WithPileup_g20*VHI_res_Reta.png}
            %\caption{~}
        %\end{subfigure}
        %%\hfill
    %\end{figure}
%\end{frame}

%\begin{frame}{Resolutions}
    %\framesubtitle{g22 - No PileUp}
    %g22 ``*VH'' trigger levels resolutions for $Eratio$, $\eta$, $ethad$, $Rhad$ and $Reta$ from top left to bottom right.
    %\begin{figure}[h!]
        %\centering
        %\begin{subfigure}[b]{.3\textwidth}
            %\centering
            %\includegraphics[width=\textwidth]{../TrigMonitoring/images/outputs/Res_Run 3 NoPileup_g22*VH_res_eratio.png}
            %\caption{~}
        %\end{subfigure}
        %%\hfill
        %\begin{subfigure}[b]{.3\textwidth}
            %\centering
            %\includegraphics[width=\textwidth]{../TrigMonitoring/images/outputs/Res_Run 3 NoPileup_g22*VH_res_eta.png}
            %\caption{~}
        %\end{subfigure}
        %%\hfill
        %\begin{subfigure}[b]{.3\textwidth}
            %\centering
            %\includegraphics[width=\textwidth]{../TrigMonitoring/images/outputs/Res_Run 3 NoPileup_g22*VH_res_ethad.png}
            %\caption{~}
        %\end{subfigure}
        %%\hfill
        %\begin{subfigure}[b]{.3\textwidth}
            %\centering
            %\includegraphics[width=\textwidth]{../TrigMonitoring/images/outputs/Res_Run 3 NoPileup_g22*VH_res_Rhad.png}
            %\caption{~}
        %\end{subfigure}
        %%\hfill
        %\begin{subfigure}[b]{.3\textwidth}
            %\centering
            %\includegraphics[width=\textwidth]{../TrigMonitoring/images/outputs/Res_Run 3 NoPileup_g22*VH_res_Reta.png}
            %\caption{~}
        %\end{subfigure}
        %%\hfill
    %\end{figure}
%\end{frame}

%\begin{frame}{Resolutions}
    %\framesubtitle{g22 - With PileUp}
    %g22 ``*VH'' trigger levels resolutions for $Eratio$, $\eta$, $ethad$, $Rhad$ and $Reta$ from top left to bottom right.
    %\begin{figure}[h!]
        %\centering
        %\begin{subfigure}[b]{.3\textwidth}
            %\centering
            %\includegraphics[width=\textwidth]{../TrigMonitoring/images/outputs/Res_Run 3 WithPileup_g22*VH_res_eratio.png}
            %\caption{~}
        %\end{subfigure}
        %%\hfill
        %\begin{subfigure}[b]{.3\textwidth}
            %\centering
            %\includegraphics[width=\textwidth]{../TrigMonitoring/images/outputs/Res_Run 3 WithPileup_g22*VH_res_eta.png}
            %\caption{~}
        %\end{subfigure}
        %%\hfill
        %\begin{subfigure}[b]{.3\textwidth}
            %\centering
            %\includegraphics[width=\textwidth]{../TrigMonitoring/images/outputs/Res_Run 3 WithPileup_g22*VH_res_ethad.png}
            %\caption{~}
        %\end{subfigure}
        %%\hfill
        %\begin{subfigure}[b]{.3\textwidth}
            %\centering
            %\includegraphics[width=\textwidth]{../TrigMonitoring/images/outputs/Res_Run 3 WithPileup_g22*VH_res_Rhad.png}
            %\caption{~}
        %\end{subfigure}
        %%\hfill
        %\begin{subfigure}[b]{.3\textwidth}
            %\centering
            %\includegraphics[width=\textwidth]{../TrigMonitoring/images/outputs/Res_Run 3 WithPileup_g22*VH_res_Reta.png}
            %\caption{~}
        %\end{subfigure}
        %%\hfill
    %\end{figure}
%\end{frame}

% TODO add ongoing and next steps as blocks
\section{More updates}
\begin{frame}
    \begin{block}{On going}
        \begin{itemize}
            \item [-]
                We got the same sample used for CNN photon Id implementation but using the HLT photon container at the presicion step,
                instead of the offline container.
            \item [-]
                We will compare the CNN Photon Id implementation against both datasets,
                offline and HLT@presicion photons,
                with the architecture of Offline CNN and HLT@Fast CNN.
        \end{itemize}
    \end{block}
    \begin{block}{Next steps}
    \begin{itemize}
        \item [-]
            Find out the causes for the low efficiencies in the icalovloose id.
        \item [-]
            Show all level efficiencies using the fixed signal samples for Run3,
            with the addition of the correctly ported CutBased selection for Fast photons.
        \item [-]
            Prepare Python script to run a set of preselected comparisons.
    \end{itemize}
    \end{block}
\end{frame}

\section{Summary}
\begin{frame}{Summary}
    \begin{itemize}
        \item [-]
            Compared Run2 triggers efficiencies against Run3 at HLT step,
            and showed that the setup and tools are ready to compare agains all other steps.\
            However, we see that the performance is generally worse than Run2 in Run3.
        \item [-]
            Showed current Run3 resolutions,
            noting that there seems to be an issue with $R_\eta$,
            and that $E_T$ is not filled.
            
        %\item 
            %We are able to \alert{train over a large sample} using \textit{generator training} (CPU only, not multithreading yet).
            %Some bugs to address still.
        %\item 
            %Current prototype CNN implementation performs better than current cut base FastReco implementation both in \textit{Inclusive} and \textit{Binned} cases,
            %in all $(\eta;p_T)$ ranges.
            %%The \textit{Inclusive Testing} shows a much \alert{better performance of the CNN model over the HLT Fast-Reco WP}, 
            %%as well as over the \textit{Offline Tight} WP (computed for reference only).
        %%\item 
            %%The \textit{Binned Testing} shows that the \alert{CNN model preforms better than the HLT Fast-Reco WP in all $(\eta;p_T)$ ranges},
        %\item
            %We can see that when feeding (or using) only the 2nd CaloLayer data,
            %the $\eta$ regions where the cells shape changes,
            %has a direct impact on performance.
            %May need one model for Barrel and other for Endcap?
    \end{itemize}
\end{frame}

\begin{frame}{Backup}
\end{frame}
\begin{frame}{Extra Slides}
\framesubtitle{g120 Run2 vs Run3 Trigger Resolutions at HLT step}
    g120 trigger levels resolutions for $Eratio$, $ethad$, $Rhad$ and $Reta$, $\eta$, $\phi$ from top left to bottom right.
    \begin{figure}[h!]
        \centering
        \begin{subfigure}[b]{.3\textwidth}
            \centering
            \includegraphics[width=\textwidth]{../TrigMonitoring/images/outputs/Run2vsRun3/Res_g120_loose_res_eratio.png}
            \caption{~}
        \end{subfigure}
        %\hfill
        \begin{subfigure}[b]{.3\textwidth}
            \centering
            \includegraphics[width=\textwidth]{../TrigMonitoring/images/outputs/Run2vsRun3/Res_g120_loose_res_ethad.png}
            \caption{~}
        \end{subfigure}
        %\hfill
        \begin{subfigure}[b]{.3\textwidth}
            \centering
            \includegraphics[width=\textwidth]{../TrigMonitoring/images/outputs/Run2vsRun3/Res_g120_loose_res_Rhad.png}
            \caption{~}
        \end{subfigure}
        %\hfill
        \begin{subfigure}[b]{.3\textwidth}
            \centering
            \includegraphics[width=\textwidth]{../TrigMonitoring/images/outputs/Run2vsRun3/Res_g120_loose_res_Reta.png}
            \caption{~}
        \end{subfigure}
        %\hfill
        \begin{subfigure}[b]{.3\textwidth}
            \centering
            \includegraphics[width=\textwidth]{../TrigMonitoring/images/outputs/Run2vsRun3/Res_g120_loose_res_eta.png}
            \caption{~}
        \end{subfigure}
        %\hfill
        \begin{subfigure}[b]{.3\textwidth}
            \centering
            \includegraphics[width=\textwidth]{../TrigMonitoring/images/outputs/Run2vsRun3/Res_g120_loose_res_phi.png}
            \caption{~}
        \end{subfigure}
    \end{figure}
\end{frame}

\begin{frame}{Extra Slides}
\framesubtitle{g25 Run2 vs Run3 Trigger Resolutions at HLT step}
    g25 trigger levels resolutions for $Eratio$, $ethad$, $Rhad$ and $Reta$, $\eta$, $\phi$ from top left to bottom right.
    \begin{figure}[h!]
        \centering
        \begin{subfigure}[b]{.3\textwidth}
            \centering
            \includegraphics[width=\textwidth]{../TrigMonitoring/images/outputs/Run2vsRun3/Res_g25_medium_res_eratio.png}
            \caption{~}
        \end{subfigure}
        %\hfill
        \begin{subfigure}[b]{.3\textwidth}
            \centering
            \includegraphics[width=\textwidth]{../TrigMonitoring/images/outputs/Run2vsRun3/Res_g25_medium_res_ethad.png}
            \caption{~}
        \end{subfigure}
        %\hfill
        \begin{subfigure}[b]{.3\textwidth}
            \centering
            \includegraphics[width=\textwidth]{../TrigMonitoring/images/outputs/Run2vsRun3/Res_g25_medium_res_Rhad.png}
            \caption{~}
        \end{subfigure}
        %\hfill
        \begin{subfigure}[b]{.3\textwidth}
            \centering
            \includegraphics[width=\textwidth]{../TrigMonitoring/images/outputs/Run2vsRun3/Res_g25_medium_res_Reta.png}
            \caption{~}
        \end{subfigure}
        %\hfill
        \begin{subfigure}[b]{.3\textwidth}
            \centering
            \includegraphics[width=\textwidth]{../TrigMonitoring/images/outputs/Run2vsRun3/Res_g25_medium_res_eta.png}
            \caption{~}
        \end{subfigure}
        %\hfill
        \begin{subfigure}[b]{.3\textwidth}
            \centering
            \includegraphics[width=\textwidth]{../TrigMonitoring/images/outputs/Run2vsRun3/Res_g25_medium_res_phi.png}
            \caption{~}
        \end{subfigure}
    \end{figure}
\end{frame}

%\begin{frame}{Extra Slides}
    %\framesubtitle{Binned ROC curves}
    %\begin{figure}[h!]
    %\label{fig:ROC1}
        %\centering
        %\begin{subfigure}[b]{.35\textwidth}
            %\centering
            %\includegraphics[width=\textwidth]{images/20201201_48x2CL_128x1DL_0.30dr_150Ep_1.0e-03lr_Generator/Val_output/plots/eta0.0xpT10.0_outROC.png}
            %\caption{~}
            %\label{fig:ROC0010}
        %\end{subfigure}
        %\hfill
        %\begin{subfigure}[b]{.35\textwidth}
            %\centering
            %\includegraphics[width=\textwidth]{images/20201201_48x2CL_128x1DL_0.30dr_150Ep_1.0e-03lr_Generator/Val_output/plots/eta0.0xpT20.0_outROC.png}
            %\caption{~}
            %\label{fig:ROC0020}
        %\end{subfigure}
        %\hfill
        %\begin{subfigure}[b]{.35\textwidth}
            %\centering
            %\includegraphics[width=\textwidth]{images/20201201_48x2CL_128x1DL_0.30dr_150Ep_1.0e-03lr_Generator/Val_output/plots/eta0.0xpT30.0_outROC.png}
            %\caption{~}
            %\label{fig:ROC0030}
        %\end{subfigure}
        %\caption{%
                %\textit{%
                    %ROC curve for 
                    %$(\eta,p_T) = (0.0,10)$~\ref{fig:ROC0010}, 
                    %$(\eta,p_T) = (0.0,20)$~\ref{fig:ROC0020}, 
                    %$(\eta,p_T) = (0.0,30)$~\ref{fig:ROC0030}, 
                    %Various WP are also ploted alongside,
                    %Tight in blue and HLT at FastReco in green.
                    %}
                %}
    %\end{figure}
%\end{frame}

%\begin{frame}{Update}
    %\begin{itemize}
        %\item 
            %I ran the TrigMonitoring tool:
            %\tiny{%
                %\texttt{%
                    %Run3DQTestingDriver.py --dqOffByDefault Input.Files=``['AOD.root']'' DQ.Steering.doHLTMon=True DQ.Steering.HLT.doJet=False DQ.Steering.HLT.doGeneral=False DQ.Steering.HLT.doEgamma=True DQ.Steering.HLT.doMET=False DQ.Steering.HLT.doBjet=False DQ.Steering.HLT.doCalo=False DQ.Steering.HLT.doMuon=False DQ.Steering.HLT.doBphys=False DQ.Steering.HLT.doMinBias=False
                %}
            %}

            %over this file (as an example):

            %\tiny{%
                %\texttt{valid1.343981.PowhegPythia8EvtGen\_NNLOPS\_nnlo\_30\_ggH125\_gamgam.recon.AOD.e5607\_e5984\_s3126\_d1605\_r12262\_tid23444304\_00/\\AOD.23444304.\_000178.pool.root.1}
            %}

        %\item
            %\textbf{WIP}
            %I am developing a SWAN notebook to visualize the efficiencies at a glance.
            %At the current state,
            %the notebook allows to explore a file,
            %and plot many histograms for a given filter criteria.
            %For instance, to search for all distributions at HLT it will be:
            %\texttt{%
                %filterpath = "run\_310000/HLT/EgammaMon/*/Distributions/HLT/*"
            %}
            %%\texttt{filterpath = }            

        %\item 
            %photon trigger levels looking at: g20, g22, g25
    %\end{itemize}
%\end{frame}

%\begin{frame}{Context}
    %\framesubtitle{The aim}
    %\begin{columns}
    %\column{0.6\textwidth}
    %\begin{itemize}
        %\item 
            %At HLT reconstruction and identification is done in an L1EM ROI and in 2 steps:
            %\begin{itemize}
                %\item 
                    %Fast step (early background rejection)
                %\item 
                    %Precision step (instantiating offline algorithms)
            %\end{itemize}
        %\item 
            %To train a CNN model to have a higher early better background rejection at the same signal efficiency (perform better) than Fast Reconstruction Cut-Based selection at photon ID.
        %\item 
            %The proposal is to use the energy distribution deposited by photons at the 2nd layer of the Calorimeter cells,
            %as images to feed a Convolutional Neural Network classifier.
        %\item 
            %Previous work related to this one were carried out by \textit{Mohamed Belfkir} for offline photon Id.
        %\item 
            %All development is carried out with MC samples.
        %\item 
            %Cells used are the ones built with the offline algorithm
            %(which are in principle different from the ones at HLT-FastReco).
    %\end{itemize}
    %\column{0.4\textwidth}
    %\begin{figure}[h!]
        %\centering
        %\includegraphics[width = 0.65\textwidth]{images/PhotonSequence.pdf}
        %\caption{%
            %\textit{%
            %Algorithm sequence at HLT for photons.
            %}
        %}
        %\label{fig:HLTalgo}
    %\end{figure}
    %\end{columns}
%\end{frame}

%\begin{frame}{Package run example}
    %\framesubtitle{Sample details}
    %Todays update will show the training and testing on the following samples.
    %\begin{itemize}
        %\item 
            %MC Data from \href{https://gitlab.cern.ch/mobelfki/zllyathderivation}{ZllyAthDerivation},
            %by Pythia8.
        %\item 
            %\textbf{DiJet} \textbf{\textit{for background only}} (14 Gb),
            %MC16 list contains: JF17, JF23, JF35 and JF50
        %\item 
            %\textbf{GammaJet} for y+jet \textbf{\textit{for signal only}} (33 Gb),
            %MC16 list contains: DP8\_17, DP17\_35, DP35\_50, DP50\_70, DP70\_140 and DP140\_280 
        %\item
            %Signal is TruthMatch,
            %Background is NonTruthMatch.
            %These define the \texttt{Labels} or classes.
        %\item 
            %\texttt{Input Data} are the Cluster Images from the \textit{\textbf{2nd Calorimeter Layer}}.
        %\item 
            %Events where selected using Strategy 1,
            %meaning only keeping those with \textbf{healthy} clusters (not missing cells) and 
            %$\eta < 2.5$.
    %\end{itemize}
    %Overall we are dealing with $47$ $Gb$ of data,
    %around $4,7 \times 10^6$ images (events) in total.
%\end{frame}

%\begin{frame}{Package run example}
    %\framesubtitle{Preprocessing details}
    %For the rest of the talk everything is computed using 
    %\href{https://gitlab.cern.ch/snoaccor/cnn-photon-id}{cnn-photon-id} 
    %package\footnote{%
    %I am pointing to my fork,
    %because the generator training was not yet sent to be merged.}
    %in \texttt{usage=HLT}.
    %\begin{itemize}
        %\item 
            %We have 3 independent subsamples (fraction of total):
            %Train ($50\%$), Validate ($25\%$), Test ($25\%$).
            %Where Train and Validate are used in the \textit{training stage},
            %and Test in the \textit{testing stage}.
        %\item
            %From the \textit{xAODs} we store the \textit{tagged images} in NumpyArrays binary files\footnote{%
           %It is actually one ordered file for each CaloLayerImage and one for the Labels}.
            %%for the Train and Validate subsamples.
            %%We refer to the set (1 Signal file, 1 Background file) as a \textit{chunk},
            %%in this case we make a total of 14.
    %\end{itemize}
    %\begin{figure}[h!]
        %\centering
        %\includegraphics[width = 0.5\textwidth]{images/SamplesDist.png}
        %\caption{%
            %\textit{%
                %Subsamples $p_T$ distributions for \textit{Training} (blue),
                %\textit{Validation} (red) and \textit{Testing} (green).
            %}
        %}
        %\label{fig:sampledist}
    %\end{figure}
%\end{frame}

%\begin{frame}{Package run example}
    %\framesubtitle{Training}
    %The training was done in two steps:
    %\begin{itemize}
        %\item[1.]
            %Dynamic model search using \href{https://keras-team.github.io/keras-tuner/}{KerasTuner} ,
            %over a subsample (about 100k images).
            %A ``\textit{dynamic search}'' means that we iterate many trainings changing the model architectures hyperparameters to evaluate the performance.
            %This allows to test different architectures over a subsample,
            %to choose the one that has the greater chances on discriminating signal and background.
            %\begin{itemize}
                %%\item 
                    %%We use a subsample of the full sample
                    %%($\simeq 2$ Gb of data about 100k images)
                %\item[1.1] 
                    %Using KerasTuner we search an Hyperparameters space,
                    %changing for instance the number of layers, nodes, etc.
                %\item[1.2]
                    %We keep the top 4 ones according to some performance criteria.
                %\item    
                    %This procedure can be quite time and resources consuming.
            %\end{itemize}
        %\item[2.]
            %We retrain those (or the best) with the full train dataset.
            %%using \textit{Python Generators}.
    %\end{itemize}
%\end{frame}

%\begin{frame}{Best model architecture}
    %\begin{columns}
    %\column{0.5\textwidth}
    %Training is done inclusive in $\eta$ and $p_T$ with the following parameters.
    %\begin{itemize}
        %\item 
            %loss: \textit{binary cross-entropy} 
        %\item
            %optimizer: Adam
        %\item 
            %learning rate: $10^{-3}$ (default keras value)
        %\item 
            %epochs: $150$
        %\item 
            %batch size: $128$
        %\item
            %the samples are shuffled and we use class re-weighting.
    %\end{itemize}
    
    %\column{0.5\textwidth}
        %\begin{figure}[h!]
        %\label{fig:model_arch}
            %\centering
            %\includegraphics[width=8cm, height=7cm, keepaspectratio]{images/20201201_48x2CL_128x1DL_0.30dr_150Ep_1.0e-03lr_Generator/model.png}
            %\caption{%
                %\textit{%
                %Model architecture.
                %}
            %}
        %\end{figure}
    %\end{columns}
%\end{frame}

%\begin{frame}{Full training results - \textbf{preliminary}}
    %\begin{figure}[h!]
        %\centering
        %\begin{subfigure}[b]{.3\textwidth}
            %\centering
            %\includegraphics[width=\textwidth]{images/20201201_48x2CL_128x1DL_0.30dr_150Ep_1.0e-03lr_Generator/history/lossvsEp.png}
            %\caption{Loss}
            %\label{subfig:LOSS}
        %\end{subfigure}
        %\begin{subfigure}[b]{.3\textwidth}
            %\centering
            %\includegraphics[width=\textwidth]{images/20201201_48x2CL_128x1DL_0.30dr_150Ep_1.0e-03lr_Generator/history/aucvsEp.png}
            %\caption{AUC}
            %\label{subfig:AUC}
        %\end{subfigure}
        %\begin{subfigure}[b]{.3\textwidth}
            %\centering
            %\includegraphics[width=\textwidth]{images/20201201_48x2CL_128x1DL_0.30dr_150Ep_1.0e-03lr_Generator/history/recallvsEp.png}
            %\caption{Signal eff}
            %\label{subfig:RECALL}
        %\end{subfigure}
        %\caption{~}
        %%\caption{%
            %%\textit{%
                %%Metrics computed during training over the whole sample (\textbf{PRELIMINARY PLOTS}).
                %%Loss~\ref{subfig:LOSS} (top left),
                %%AUC~\ref{subfig:AUC} (top right)
                %%and signal efficiency or recall~\ref{subfig:RECALL} (bottom),
                %%as metrics $vs$ epochs for training (blue) and validation (red). 
                %%}
            %%}
        %\label{fig:metrics}
    %\end{figure}
    %\begin{itemize}
        %\item
            %Training truncated early $\rightarrow$ \alert{unexpected}.
            %Under investigation (bug in the generator loop?).
        %\item
            %Looking at both the AUC and the Loss, 
            %we can see that the best set of weights is reached at epoch 11.
            %After that we see divergence of the fitting,
            %probably related to the former point
            %(issue with the optimizer? the learning rate? the memory?)
        %\item
            %\centering
            %\begin{tabular}{l|c|c|c|}
                %Epoch    &Loss   &$\text{AUC}_{val}$ &$\text{Recall}_{val}$ \\
                %\hline
                %$11$     &$0.25$ &$0.9455$           &$0.887$
            %\end{tabular} 
    %\end{itemize}
%\end{frame}

%\begin{frame}{Full testing - \textbf{preliminary}}
    %\framesubtitle{Inclusive ROC curve and performance comparison}
    %\begin{itemize}
        %\item
            %The testing is carried over the whole subsample \textit{testing} using only 1 chunk.
        %\item
            %The testing is computed both \textbf{\textit{Inclusive}} and \textbf{\textit{Binned}}.
        %\item 
            %The ROC curve for the trained model,
            %carried over the inclusive testing subsample is plotted in orange.
        %\item 
            %Along side this curve the FastReco selection WP (green circle)
            %and the Tight WP (blue circle) are plotted.
    %\end{itemize}
    %\begin{figure}[h!]
        %\centering
        %\includegraphics[width=.7\textwidth]{images/20201201_48x2CL_128x1DL_0.30dr_150Ep_1.0e-03lr_Generator/Val_output/plots/full_outROC.png}
        %\caption{~}
        %\label{fig:ROCInc}
    %\end{figure}
%\end{frame}

%\begin{frame}{Binned testing results - \textbf{preliminary}}
    %\begin{itemize}
        %\item 
            %AUC matrix for MC data in Testing over the full \textit{Test Sample}.
        %\item 
            %Each matrix elements corresponds to the integrated ROC curve for the respective $(\eta,p_T)$ range.
        %\item 
            %The uncertainties are all statistical,
            %and computed integrating the lower (upper)``error'' ROC curves,
            %more info \href{https://gitlab.cern.ch/snoaccor/roc_witherrorbars}{here}.
        %\item 
            %Comparison against CutBased $(\eta,p_T)$ in backup.
    %\end{itemize}
    %\begin{figure}[h!]
        %\centering
        %\includegraphics[width=.75\textwidth]{images/20201201_48x2CL_128x1DL_0.30dr_150Ep_1.0e-03lr_Generator/Val_output/plots/AUC_Mode2_HEATMAP.png}
        %\caption{~}
            %%\textit{%
                %%AUC matrix for MC data in Validation over the full \textit{Test Sample}.
                %%Each matrix elements corresponds to the integrated ROC curve for the respective 
                %%$(\eta,p_T)$ range.
                %%The uncertainties are all statistical,
                %%and computed integrating the lower (upper)``error'' ROC curves,
                %%more info \href{https://gitlab.cern.ch/snoaccor/roc_witherrorbars}{here}.
            %%}
        %%}
        %\label{fig:AUCMAT}
    %\end{figure}
%\end{frame}
%\begin{frame}{Extra Slides}
    %\framesubtitle{Binned ROC curves}
    %\begin{figure}[h!]
    %\label{fig:ROC2}
        %\centering
        %\begin{subfigure}[b]{.35\textwidth}
            %\centering
            %\includegraphics[width=\textwidth]{images/20201201_48x2CL_128x1DL_0.30dr_150Ep_1.0e-03lr_Generator/Val_output/plots/eta0.0xpT40.0_outROC.png}
            %\caption{~}
            %\label{fig:ROC0040}
        %\end{subfigure}
        %\hfill
        %\begin{subfigure}[b]{.35\textwidth}
            %\centering
            %\includegraphics[width=\textwidth]{images/20201201_48x2CL_128x1DL_0.30dr_150Ep_1.0e-03lr_Generator/Val_output/plots/eta0.0xpT60.0_outROC.png}
            %\caption{~}
            %\label{fig:ROC0060}
        %\end{subfigure}
        %\hfill
        %\begin{subfigure}[b]{.35\textwidth}
            %\centering
            %\includegraphics[width=\textwidth]{images/20201201_48x2CL_128x1DL_0.30dr_150Ep_1.0e-03lr_Generator/Val_output/plots/eta0.0xpT80.0_outROC.png}
            %\caption{~}
            %\label{fig:ROC0080}
        %\end{subfigure}
        %\caption{%
                %\textit{%
                    %ROC curve for 
                    %$(\eta,p_T) = (0.0,40)$~\ref{fig:ROC0040}, 
                    %$(\eta,p_T) = (0.0,60)$~\ref{fig:ROC0060}, 
                    %$(\eta,p_T) = (0.0,80)$~\ref{fig:ROC0080}, 
                    %Various WP are also ploted alongside,
                    %Tight in blue and HLT at FastReco in green.
                    %}
                %}
    %\end{figure}
%\end{frame}

%\begin{frame}{Extra Slides}
    %\framesubtitle{Binned ROC curves}
    %\begin{figure}[h!]
    %\label{fig:ROC3}
        %\centering
        %\begin{subfigure}[b]{.35\textwidth}
            %\centering
            %\includegraphics[width=\textwidth]{images/20201201_48x2CL_128x1DL_0.30dr_150Ep_1.0e-03lr_Generator/Val_output/plots/eta0.6xpT10.0_outROC.png}
            %\caption{~}
            %\label{fig:ROC0610}
        %\end{subfigure}
        %\hfill
        %\begin{subfigure}[b]{.35\textwidth}
            %\centering
            %\includegraphics[width=\textwidth]{images/20201201_48x2CL_128x1DL_0.30dr_150Ep_1.0e-03lr_Generator/Val_output/plots/eta0.6xpT20.0_outROC.png}
            %\caption{~}
            %\label{fig:ROC0620}
        %\end{subfigure}
        %\hfill
        %\begin{subfigure}[b]{.35\textwidth}
            %\centering
            %\includegraphics[width=\textwidth]{images/20201201_48x2CL_128x1DL_0.30dr_150Ep_1.0e-03lr_Generator/Val_output/plots/eta0.6xpT30.0_outROC.png}
            %\caption{~}
            %\label{fig:ROC0630}
        %\end{subfigure}
        %\caption{%
                %\textit{%
                    %ROC curve for 
                    %$(\eta,p_T) = (0.6,10)$~\ref{fig:ROC0610}, 
                    %$(\eta,p_T) = (0.6,20)$~\ref{fig:ROC0620}, 
                    %$(\eta,p_T) = (0.6,30)$~\ref{fig:ROC0630}, 
                    %Various WP are also ploted alongside,
                    %Tight in blue and HLT at FastReco in green.
                    %}
                %}
    %\end{figure}
%\end{frame}

%\begin{frame}{Extra Slides}
    %\framesubtitle{Binned ROC curves}
    %\begin{figure}[h!]
        %\centering
        %\begin{subfigure}[b]{.35\textwidth}
            %\centering
            %\includegraphics[width=\textwidth]{images/20201201_48x2CL_128x1DL_0.30dr_150Ep_1.0e-03lr_Generator/Val_output/plots/eta0.6xpT40.0_outROC.png}
            %\caption{~}
            %\label{fig:ROC0640}
        %\end{subfigure}
        %\hfill
        %\begin{subfigure}[b]{.35\textwidth}
            %\centering
            %\includegraphics[width=\textwidth]{images/20201201_48x2CL_128x1DL_0.30dr_150Ep_1.0e-03lr_Generator/Val_output/plots/eta0.6xpT60.0_outROC.png}
            %\caption{~}
            %\label{fig:ROC0660}
        %\end{subfigure}
        %\hfill
        %\begin{subfigure}[b]{.35\textwidth}
            %\centering
            %\includegraphics[width=\textwidth]{images/20201201_48x2CL_128x1DL_0.30dr_150Ep_1.0e-03lr_Generator/Val_output/plots/eta0.6xpT80.0_outROC.png}
            %\caption{~}
            %\label{fig:ROC0680}
        %\end{subfigure}
        %\caption{%
                %\textit{%
                    %ROC curve for 
                    %$(\eta,p_T) = (0.6,40)$~\ref{fig:ROC0640}, 
                    %$(\eta,p_T) = (0.6,60)$~\ref{fig:ROC0660}, 
                    %$(\eta,p_T) = (0.6,80)$~\ref{fig:ROC0680}, 
                    %Various WP are also ploted alongside,
                    %Tight in blue and HLT at FastReco in green.
                    %}
                %}
        %\label{fig:ROC4}
    %\end{figure}
%\end{frame}

%\begin{frame}{Extra Slides}
    %\framesubtitle{Binned ROC curves}
    %\begin{figure}[h!]
    %\label{fig:ROC5}
        %\centering
        %\begin{subfigure}[b]{.35\textwidth}
            %\centering
            %\includegraphics[width=\textwidth]{images/20201201_48x2CL_128x1DL_0.30dr_150Ep_1.0e-03lr_Generator/Val_output/plots/eta0.8xpT10.0_outROC.png}
            %\caption{~}
            %\label{fig:ROC0810}
        %\end{subfigure}
        %\hfill
        %\begin{subfigure}[b]{.35\textwidth}
            %\centering
            %\includegraphics[width=\textwidth]{images/20201201_48x2CL_128x1DL_0.30dr_150Ep_1.0e-03lr_Generator/Val_output/plots/eta0.8xpT20.0_outROC.png}
            %\caption{~}
            %\label{fig:ROC0820}
        %\end{subfigure}
        %\hfill
        %\begin{subfigure}[b]{.35\textwidth}
            %\centering
            %\includegraphics[width=\textwidth]{images/20201201_48x2CL_128x1DL_0.30dr_150Ep_1.0e-03lr_Generator/Val_output/plots/eta0.8xpT30.0_outROC.png}
            %\caption{~}
            %\label{fig:ROC0830}
        %\end{subfigure}
        %\caption{%
                %\textit{%
                    %ROC curve for 
                    %$(\eta,p_T) = (0.8,10)$~\ref{fig:ROC0810}, 
                    %$(\eta,p_T) = (0.8,20)$~\ref{fig:ROC0820}, 
                    %$(\eta,p_T) = (0.8,30)$~\ref{fig:ROC0830}, 
                    %Various WP are also ploted alongside,
                    %Tight in blue and HLT at FastReco in green.
                    %}
                %}
    %\end{figure}
%\end{frame}

%\begin{frame}{Extra Slides}
    %\framesubtitle{Binned ROC curves}
    %\begin{figure}[h!]
        %\centering
        %\begin{subfigure}[b]{.35\textwidth}
            %\centering
            %\includegraphics[width=\textwidth]{images/20201201_48x2CL_128x1DL_0.30dr_150Ep_1.0e-03lr_Generator/Val_output/plots/eta0.8xpT40.0_outROC.png}
            %\caption{~}
            %\label{fig:ROC0840}
        %\end{subfigure}
        %\hfill
        %\begin{subfigure}[b]{.35\textwidth}
            %\centering
            %\includegraphics[width=\textwidth]{images/20201201_48x2CL_128x1DL_0.30dr_150Ep_1.0e-03lr_Generator/Val_output/plots/eta0.8xpT60.0_outROC.png}
            %\caption{~}
            %\label{fig:ROC0860}
        %\end{subfigure}
        %\hfill
        %\begin{subfigure}[b]{.35\textwidth}
            %\centering
            %\includegraphics[width=\textwidth]{images/20201201_48x2CL_128x1DL_0.30dr_150Ep_1.0e-03lr_Generator/Val_output/plots/eta0.8xpT80.0_outROC.png}
            %\caption{~}
            %\label{fig:ROC0880}
        %\end{subfigure}
        %\caption{%
                %\textit{%
                    %ROC curve for 
                    %$(\eta,p_T) = (0.8,40)$~\ref{fig:ROC0840}, 
                    %$(\eta,p_T) = (0.8,60)$~\ref{fig:ROC0860}, 
                    %$(\eta,p_T) = (0.8,80)$~\ref{fig:ROC0880}, 
                    %Various WP are also ploted alongside,
                    %Tight in blue and HLT at FastReco in green.
                    %}
                %}
        %\label{fig:ROC6}
    %\end{figure}
%\end{frame}

%\begin{frame}{Extra Slides}
    %\framesubtitle{Binned ROC curves}
    %\begin{figure}[h!]
        %\centering
        %\begin{subfigure}[b]{.35\textwidth}
            %\centering
            %\includegraphics[width=\textwidth]{images/20201201_48x2CL_128x1DL_0.30dr_150Ep_1.0e-03lr_Generator/Val_output/plots/eta1.1xpT10.0_outROC.png}
            %\caption{~}
            %\label{fig:ROC1110}
        %\end{subfigure}
        %\hfill
        %\begin{subfigure}[b]{.35\textwidth}
            %\centering
            %\includegraphics[width=\textwidth]{images/20201201_48x2CL_128x1DL_0.30dr_150Ep_1.0e-03lr_Generator/Val_output/plots/eta1.1xpT20.0_outROC.png}
            %\caption{~}
            %\label{fig:ROC1120}
        %\end{subfigure}
        %\hfill
        %\begin{subfigure}[b]{.35\textwidth}
            %\centering
            %\includegraphics[width=\textwidth]{images/20201201_48x2CL_128x1DL_0.30dr_150Ep_1.0e-03lr_Generator/Val_output/plots/eta1.1xpT30.0_outROC.png}
            %\caption{~}
            %\label{fig:ROC1130}
        %\end{subfigure}
        %\caption{%
                %\textit{%
                    %ROC curve for 
                    %$(\eta,p_T) = (1.1,10)$~\ref{fig:ROC1110}, 
                    %$(\eta,p_T) = (1.1,20)$~\ref{fig:ROC1120}, 
                    %$(\eta,p_T) = (1.1,30)$~\ref{fig:ROC1130}, 
                    %Various WP are also ploted alongside,
                    %Tight in blue and HLT at FastReco in green.
                    %}
                %}
        %\label{fig:ROC7}
    %\end{figure}
%\end{frame}

%\begin{frame}{Extra Slides}
    %\framesubtitle{Binned ROC curves}
    %\begin{figure}[h!]
        %\centering
        %\begin{subfigure}[b]{.35\textwidth}
            %\centering
            %\includegraphics[width=\textwidth]{images/20201201_48x2CL_128x1DL_0.30dr_150Ep_1.0e-03lr_Generator/Val_output/plots/eta1.1xpT40.0_outROC.png}
            %\caption{~}
            %\label{fig:ROC1140}
        %\end{subfigure}
        %\hfill
        %\begin{subfigure}[b]{.35\textwidth}
            %\centering
            %\includegraphics[width=\textwidth]{images/20201201_48x2CL_128x1DL_0.30dr_150Ep_1.0e-03lr_Generator/Val_output/plots/eta1.1xpT60.0_outROC.png}
            %\caption{~}
            %\label{fig:ROC1160}
        %\end{subfigure}
        %\hfill
        %\begin{subfigure}[b]{.35\textwidth}
            %\centering
            %\includegraphics[width=\textwidth]{images/20201201_48x2CL_128x1DL_0.30dr_150Ep_1.0e-03lr_Generator/Val_output/plots/eta1.1xpT80.0_outROC.png}
            %\caption{~}
            %\label{fig:ROC1180}
        %\end{subfigure}
        %\caption{%
                %\textit{%
                    %ROC curve for 
                    %$(\eta,p_T) = (1.1,40)$~\ref{fig:ROC1140}, 
                    %$(\eta,p_T) = (1.1,60)$~\ref{fig:ROC1160}, 
                    %$(\eta,p_T) = (1.1,80)$~\ref{fig:ROC1180}, 
                    %Various WP are also ploted alongside,
                    %Tight in blue and HLT at FastReco in green.
                    %}
                %}
        %\label{fig:ROC8}
    %\end{figure}
%\end{frame}

%\begin{frame}{Extra Slides}
    %\framesubtitle{Binned ROC curves}
    %\begin{figure}[h!]
        %\centering
        %\begin{subfigure}[b]{.35\textwidth}
            %\centering
            %\includegraphics[width=\textwidth]{images/20201201_48x2CL_128x1DL_0.30dr_150Ep_1.0e-03lr_Generator/Val_output/plots/eta1.5xpT10.0_outROC.png}
            %\caption{~}
            %\label{fig:ROC1510}
        %\end{subfigure}
        %\hfill
        %\begin{subfigure}[b]{.35\textwidth}
            %\centering
            %\includegraphics[width=\textwidth]{images/20201201_48x2CL_128x1DL_0.30dr_150Ep_1.0e-03lr_Generator/Val_output/plots/eta1.5xpT20.0_outROC.png}
            %\caption{~}
            %\label{fig:ROC1520}
        %\end{subfigure}
        %\hfill
        %\begin{subfigure}[b]{.35\textwidth}
            %\centering
            %\includegraphics[width=\textwidth]{images/20201201_48x2CL_128x1DL_0.30dr_150Ep_1.0e-03lr_Generator/Val_output/plots/eta1.5xpT30.0_outROC.png}
            %\caption{~}
            %\label{fig:ROC1530}
        %\end{subfigure}
        %\caption{%
                %\textit{%
                    %ROC curve for 
                    %$(\eta,p_T) = (1.5,10)$~\ref{fig:ROC1510}, 
                    %$(\eta,p_T) = (1.5,20)$~\ref{fig:ROC1520}, 
                    %$(\eta,p_T) = (1.5,30)$~\ref{fig:ROC1530}, 
                    %Various WP are also ploted alongside,
                    %Tight in blue and HLT at FastReco in green.
                    %}
                %}
    %\label{fig:ROC9}
    %\end{figure}
%\end{frame}

%\begin{frame}{Extra Slides}
    %\framesubtitle{Binned ROC curves}
    %\begin{figure}[h!]
        %\centering
        %\begin{subfigure}[b]{.35\textwidth}
            %\centering
            %\includegraphics[width=\textwidth]{images/20201201_48x2CL_128x1DL_0.30dr_150Ep_1.0e-03lr_Generator/Val_output/plots/eta1.5xpT40.0_outROC.png}
            %\caption{~}
            %\label{fig:ROC1540}
        %\end{subfigure}
        %\hfill
        %\begin{subfigure}[b]{.35\textwidth}
            %\centering
            %\includegraphics[width=\textwidth]{images/20201201_48x2CL_128x1DL_0.30dr_150Ep_1.0e-03lr_Generator/Val_output/plots/eta1.5xpT60.0_outROC.png}
            %\caption{~}
            %\label{fig:ROC1560}
        %\end{subfigure}
        %\hfill
        %\begin{subfigure}[b]{.35\textwidth}
            %\centering
            %\includegraphics[width=\textwidth]{images/20201201_48x2CL_128x1DL_0.30dr_150Ep_1.0e-03lr_Generator/Val_output/plots/eta1.5xpT80.0_outROC.png}
            %\caption{~}
            %\label{fig:ROC1580}
        %\end{subfigure}
        %\caption{%
                %\textit{%
                    %ROC curve for 
                    %$(\eta,p_T) = (1.5,40)$~\ref{fig:ROC1540}, 
                    %$(\eta,p_T) = (1.5,60)$~\ref{fig:ROC1560}, 
                    %$(\eta,p_T) = (1.5,80)$~\ref{fig:ROC1580}, 
                    %Various WP are also ploted alongside,
                    %Tight in blue and HLT at FastReco in green.
                    %}
                %}
    %\label{fig:ROC10}
    %\end{figure}
%\end{frame}

%\begin{frame}{Extra Slides}
    %\framesubtitle{Binned ROC curves}
    %\begin{figure}[h!]
    %\label{fig:ROC11}
        %\centering
        %\begin{subfigure}[b]{.35\textwidth}
            %\centering
            %\includegraphics[width=\textwidth]{images/20201201_48x2CL_128x1DL_0.30dr_150Ep_1.0e-03lr_Generator/Val_output/plots/eta1.8xpT10.0_outROC.png}
            %\caption{~}
            %\label{fig:ROC1810}
        %\end{subfigure}
        %\hfill
        %\begin{subfigure}[b]{.35\textwidth}
            %\centering
            %\includegraphics[width=\textwidth]{images/20201201_48x2CL_128x1DL_0.30dr_150Ep_1.0e-03lr_Generator/Val_output/plots/eta1.8xpT20.0_outROC.png}
            %\caption{~}
            %\label{fig:ROC1820}
        %\end{subfigure}
        %\hfill
        %\begin{subfigure}[b]{.35\textwidth}
            %\centering
            %\includegraphics[width=\textwidth]{images/20201201_48x2CL_128x1DL_0.30dr_150Ep_1.0e-03lr_Generator/Val_output/plots/eta1.8xpT30.0_outROC.png}
            %\caption{~}
            %\label{fig:ROC1830}
        %\end{subfigure}
        %\caption{%
                %\textit{%
                    %ROC curve for 
                    %$(\eta,p_T) = (1.8,10)$~\ref{fig:ROC1810}, 
                    %$(\eta,p_T) = (1.8,20)$~\ref{fig:ROC1820}, 
                    %$(\eta,p_T) = (1.8,30)$~\ref{fig:ROC1830}, 
                    %Various WP are also ploted alongside,
                    %Tight in blue and HLT at FastReco in green.
                    %}
                %}
    %\end{figure}
%\end{frame}

%\begin{frame}{Extra Slides}
    %\framesubtitle{Binned ROC curves}
    %\begin{figure}[h!]
    %\label{fig:ROC12}
        %\centering
        %\begin{subfigure}[b]{.35\textwidth}
            %\centering
            %\includegraphics[width=\textwidth]{images/20201201_48x2CL_128x1DL_0.30dr_150Ep_1.0e-03lr_Generator/Val_output/plots/eta1.8xpT40.0_outROC.png}
            %\caption{~}
            %\label{fig:ROC1840}
        %\end{subfigure}
        %\hfill
        %\begin{subfigure}[b]{.35\textwidth}
            %\centering
            %\includegraphics[width=\textwidth]{images/20201201_48x2CL_128x1DL_0.30dr_150Ep_1.0e-03lr_Generator/Val_output/plots/eta1.8xpT60.0_outROC.png}
            %\caption{~}
            %\label{fig:ROC1860}
        %\end{subfigure}
        %\hfill
        %\begin{subfigure}[b]{.35\textwidth}
            %\centering
            %\includegraphics[width=\textwidth]{images/20201201_48x2CL_128x1DL_0.30dr_150Ep_1.0e-03lr_Generator/Val_output/plots/eta1.8xpT80.0_outROC.png}
            %\caption{~}
            %\label{fig:ROC1880}
        %\end{subfigure}
        %\caption{%
                %\textit{%
                    %ROC curve for 
                    %$(\eta,p_T) = (1.8,40)$~\ref{fig:ROC1840}, 
                    %$(\eta,p_T) = (1.8,60)$~\ref{fig:ROC1860}, 
                    %$(\eta,p_T) = (1.8,80)$~\ref{fig:ROC1880}, 
                    %Various WP are also ploted alongside,
                    %Tight in blue and HLT at FastReco in green.
                    %}
                %}
    %\end{figure}
%\end{frame}

%\begin{frame}{Extra Slides}
    %\framesubtitle{Binned ROC curves}
    %\begin{figure}[h!]
    %\label{fig:ROC13}
        %\centering
        %\begin{subfigure}[b]{.35\textwidth}
            %\centering
            %\includegraphics[width=\textwidth]{images/20201201_48x2CL_128x1DL_0.30dr_150Ep_1.0e-03lr_Generator/Val_output/plots/eta2.0xpT10.0_outROC.png}
            %\caption{~}
            %\label{fig:ROC2010}
        %\end{subfigure}
        %\hfill
        %\begin{subfigure}[b]{.35\textwidth}
            %\centering
            %\includegraphics[width=\textwidth]{images/20201201_48x2CL_128x1DL_0.30dr_150Ep_1.0e-03lr_Generator/Val_output/plots/eta2.0xpT20.0_outROC.png}
            %\caption{~}
            %\label{fig:ROC2020}
        %\end{subfigure}
        %\hfill
        %\begin{subfigure}[b]{.35\textwidth}
            %\centering
            %\includegraphics[width=\textwidth]{images/20201201_48x2CL_128x1DL_0.30dr_150Ep_1.0e-03lr_Generator/Val_output/plots/eta2.0xpT30.0_outROC.png}
            %\caption{~}
            %\label{fig:ROC2030}
        %\end{subfigure}
        %\caption{%
                %\textit{%
                    %ROC curve for 
                    %$(\eta,p_T) = (2.0,10)$~\ref{fig:ROC2010}, 
                    %$(\eta,p_T) = (2.0,20)$~\ref{fig:ROC2020}, 
                    %$(\eta,p_T) = (2.0,30)$~\ref{fig:ROC2030}, 
                    %Various WP are also ploted alongside,
                    %Tight in blue and HLT at FastReco in green.
                    %}
                %}
    %\end{figure}
%\end{frame}

%\begin{frame}{Extra Slides}
    %\framesubtitle{Binned ROC curves}
    %\begin{figure}[h!]
    %\label{fig:ROC14}
        %\centering
        %\begin{subfigure}[b]{.35\textwidth}
            %\centering
            %\includegraphics[width=\textwidth]{images/20201201_48x2CL_128x1DL_0.30dr_150Ep_1.0e-03lr_Generator/Val_output/plots/eta2.0xpT40.0_outROC.png}
            %\caption{~}
            %\label{fig:ROC2040}
        %\end{subfigure}
        %\hfill
        %\begin{subfigure}[b]{.35\textwidth}
            %\centering
            %\includegraphics[width=\textwidth]{images/20201201_48x2CL_128x1DL_0.30dr_150Ep_1.0e-03lr_Generator/Val_output/plots/eta2.0xpT60.0_outROC.png}
            %\caption{~}
            %\label{fig:ROC2060}
        %\end{subfigure}
        %\hfill
        %\begin{subfigure}[b]{.35\textwidth}
            %\centering
            %\includegraphics[width=\textwidth]{images/20201201_48x2CL_128x1DL_0.30dr_150Ep_1.0e-03lr_Generator/Val_output/plots/eta2.0xpT80.0_outROC.png}
            %\caption{~}
            %\label{fig:ROC2080}
        %\end{subfigure}
        %\caption{%
                %\textit{%
                    %ROC curve for 
                    %$(\eta,p_T) = (2.0,40)$~\ref{fig:ROC2040}, 
                    %$(\eta,p_T) = (2.0,60)$~\ref{fig:ROC2060}, 
                    %$(\eta,p_T) = (2.0,80)$~\ref{fig:ROC2080}, 
                    %Various WP are also ploted alongside,
                    %Tight in blue and HLT at FastReco in green.
                    %}
                %}
    %\end{figure}
%\end{frame}

%\begin{frame}{Update on the package}
    %\begin{itemize}
        %\item (DONE)
            %cleaning, documenting simplifying the package, to improve both readablity and mantainability.
            %to run the whole package from a config file.
        %\item (ONGOING) 
            %made progress on testing running @grid.
            %Two arguments motivate this implementation,
            %one is the access to more resourses like RAM and CPU power that would allow to train on much bigger samples,
            %and the second one is the natural integration with ATLAS workflow and frameworks.
        %\item (ONGOING)
            %integration with ONNX.
    %\end{itemize}
%\end{frame}

%\begin{frame}
    %\frametitle{Details of the package structural changes}
    %\begin{itemize}
        %\item (new) computing of ROC-curves with uncertainties, more info \href{https://gitlab.cern.ch/snoaccor/roc_witherrorbars}{here}
    %\end{itemize}
%\end{frame}

%\begin{frame}{Samples}
    %\textbf{Data}
    %\begin{itemize}
        %\item Data from \href{https://gitlab.cern.ch/snoaccor/CellDumpingPackage}{CellDumpingPackage}(subsample to run locally).
            %Only keeping events in Barrel.
        %\item Signal is Passed ZRad selection, Background is reverse ETcone20 isolation.
    %\end{itemize}
%\end{frame}



%\begin{frame}{Preformance of model in MCData}
    %\begin{figure}[h!]
        %\centering
        %\includegraphics[width = 0.7\textwidth]{images/roc_curves/plots/ROC_Inclusive.png}
        %\caption{%
            %\textit{%
                %AUC matrix where each cell/element contains the AUC of the ROC curve that corresponds to the pair $(\eta,p_t)$.
            %Performance of the model trained on MC data,
            %}
        %}
        %\label{fig:Performance}
    %\end{figure}   	
%\end{frame}

%\begin{frame}{Preformance of model in MCData}
    %\begin{figure}[h!]
        %\centering
        %\includegraphics[width = 0.7\textwidth]{images/roc_curves/plots/full_outROC.png}
        %\caption{%
            %\textit{%
            %The inclusive ROC curve in $(\eta,p_T)$,
            %Various WP are also ploted alongside,
            %Tight in blue and HLT at FastReco in green.
            %}
        %}
        %\label{fig:Performance}
    %\end{figure}   	
%\end{frame}

%\section{Section Two}

%\begin{frame}{Slide with table}
	%\input{tables/table1.tex}
%\end{frame}


%\begin{frame}{Slide with references}
	%This is to reference a figure (Figure \ref{fig:figure1})\\
    %This it to reference a table (Table \ref{tab:table1})\\
    %This is to cite an article \cite{Ahmed2018a}\\
    %This is to add an article to the references without mentioning in the text \nocite{Ahmed2018a}\\
%\end{frame}
%\section{References}

%% Adding the option 'allowframebreaks' allows the contents of the slide to be expanded in more than one slide.
%\begin{frame}[allowframebreaks]{References}
	%\tiny\bibliography{references}
	%\bibliographystyle{apalike}
%\end{frame}

\end{document}
